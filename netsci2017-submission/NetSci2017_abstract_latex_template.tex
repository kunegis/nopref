% This is the template for submission of abstracts to NetSci 2017 in Indianapolis, IN.
% It is modified from NetSci 2016.
% The editor of the booklet reserves the right to modify your submission.

%% To process this file run LaTeX2e

%%********DO NOT EDIT****************
\documentclass[12pt]{article}
\usepackage{mathptmx}
\usepackage{graphicx}
\pagestyle{empty}

\setlength\topmargin{0pt}
\addtolength\topmargin{-\headheight}
\addtolength\topmargin{-\headsep}
\setlength\oddsidemargin{0pt}
\setlength\textwidth{\paperwidth}
\addtolength\textwidth{-2in}
\setlength\textheight{\paperheight}
\addtolength\textheight{-2in}
\usepackage{layout}

\renewcommand{\title}[1]{\noindent\textbf{#1}\bigskip\\}
\renewcommand{\author}[1]{\noindent #1\bigskip\\}
%%***********************************

\begin{document}

%**********USER DEFINED**************
%Enter title here
\title{Welcome to NetSci2017}
%Enter author(s) and address here
\author{Yong-Yeol Ahn,$^1$ Ciro Cattuto,$^2$ and Tina Eliassi-Rad$^3$\bigskip\\
{\small
1. Indiana University, Bloomington, IN, USA\\
2. ISI Foundation, Turin, Italy\\
3. Northeastern University, Boston, MA, USA
}
}
%Enter abstract here
NetSci 2017 will be held in Indianapolis, IN from June 19 to 23, 2017. School and Satellites will be held Monday, June 19 and Tuesday, June 20. Conference programming will begin on Wednesday, June 21 and continue through Friday, June 23. The NetSci conference is an annual meeting of the Network Science Society [1] and is hosted this year by the Indiana University Network Science Institute (IUNI). Conference Co-Chairs are Olaf Sporns [2] and Filippo Menczer [3]. 

\bigskip
\noindent[1] You may add a reference.
\\
\noindent[2] You may add a reference.
\\
\noindent[3] You may add a reference.
\\

\begin{figure}[!h]
\begin{center}
\includegraphics[scale=0.4]{NetSci_Logo.png}
\end{center}
\caption{NetSci Logo}
\end{figure}
% Place the abstract of your talk/poster here, 250 Words maximum.
% Mathematical formulae may be set in LaTeX, but do NOT use the
% bibliography environment -- if you must have references, use an 
% enumerated list.
%
% Your abstract (plus one figure) should not exceed one page.
%
% Check carefully.

%************************************
\end{document}